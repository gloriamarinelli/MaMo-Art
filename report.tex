\documentclass[a4paper,12pt]{article}
\usepackage{graphicx}
\usepackage{lipsum} 
\usepackage{ragged2e}
\usepackage{hyperref}
\usepackage{parskip} 

\begin{document}

\begin{titlepage}
    \begingroup
    \centering
    \includegraphics[width=0.5\textwidth]{sapienza.png}\par\vspace{1cm}

    {\bfseries\large Data Management Project \par}\vspace{0.5cm}

    {\bfseries\LARGE Mamo Art Gallery \par}\vspace{1cm}

    {\small Faculty of Information Engineering, Computer Science and Statistics
        \par}\vspace{0.1cm}

    {\small
        M.Sc. Engineering in Computer Science
        \par}\vspace{10cm}

    \endgroup

    {\textbf{\small
            Presented by: \\
            Gloria Marinelli, 2054014 \\
            Mario Morra}}\vfill

\end{titlepage}

\tableofcontents
\newpage

\section{Introduction}
\justify
The idea is a web-based application which serves as an online marketplace for the search and purchase of artworks created by a number of different artists. Potential purchasers are afforded the opportunity to explore a variety of paintings prior to making a purchase.

The application's architectural framework consists of the following components:

\begin{itemize}
    \item A database, managed by MongoDB.
    \item A backend, constructed using Python with the Flask web framework.
    \item A frontend, developed using the ReactJS framework.
\end{itemize}

The entire application is based on two datasets: the \href{https://www.kaggle.com/datasets/isaienkov/edvard-munch-paintings}{Edvard Munch Paintings dataset} and the \href{https://www.kaggle.com/datasets/momanyc/museum-collection?select=artworks.csv}{Museum of Modern Art (MoMA) Collection dataset}.

The web application is structured as follows:

\begin{itemize}
    \item The login page
    \item the sign up page

    \item The homepage: displays all available paintings at MaMo Art gallery. Selecting a painting will display a summary of relevant information about the artwork, including the title, name, date, medium, dimensions, acquisition date, and other pertinent details. Furthermore, users have the option to purchase the selected artwork directly from this page, which will then be added to their list of orders.

    \item The orders page: displays all orders placed by the user, including the Order ID, the Artwork ID and the date of the order.

    \item The artists page:  displays all artists associated with MaMo Art Gallery. Clicking on an artist displays their biography (nationality, gender, birth year, death year) and the list of their relative paintings.

\end{itemize}

\section{Dataset Description}
\justify

\section{Database: MongoDB}
\justify

\subsection{Why use MongoDB?}
\justify

\section{Backend: Flask and Python}
\justify

\section{Frontend: React}
\justify

\end{document}
